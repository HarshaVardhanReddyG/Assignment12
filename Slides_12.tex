\documentclass{beamer}
\usetheme{CambridgeUS}

\setbeamertemplate{caption}[numbered]{}

\usepackage{enumitem}
\usepackage{tfrupee}
\usepackage{amsmath}
\usepackage{amssymb}
\usepackage{textcomp, gensymb}
\usepackage{graphicx}
\usepackage{txfonts}

\def\inputGnumericTable{}

\usepackage[latin1]{inputenc}                                 
\usepackage{color}                                            
\usepackage{array}                                            
\usepackage{longtable}                                        
\usepackage{calc}                                             
\usepackage{multirow}                                         
\usepackage{hhline}                                           
\usepackage{ifthen}
\usepackage{caption} 
\providecommand{\mbf}{\mathbf}
\providecommand{\qfunc}[1]{\ensuremath{Q\left(#1\right)}}
\providecommand{\sbrak}[1]{\ensuremath{{}\left[#1\right]}}
\providecommand{\lsbrak}[1]{\ensuremath{{}\left[#1\right.}}
\providecommand{\rsbrak}[1]{\ensuremath{{}\left.#1\right]}}
\providecommand{\brak}[1]{\ensuremath{\left(#1\right)}}
\providecommand{\lbrak}[1]{\ensuremath{\left(#1\right.}}
\providecommand{\rbrak}[1]{\ensuremath{\left.#1\right)}}
\providecommand{\cbrak}[1]{\ensuremath{\left\{#1\right\}}}
\providecommand{\lcbrak}[1]{\ensuremath{\left\{#1\right.}}
\providecommand{\rcbrak}[1]{\ensuremath{\left.#1\right\}}}                                  
                               
\title{Assignment 12}
\author{G HARSHA VARDHAN REDDY ( CS21BTECH11017 )}
\date{\today}
\logo{\large{AI1110}}


\begin{document}
% Title page frame
\begin{frame}
    \titlepage 
\end{frame}

% Remove logo from the next slides
\logo{}


% Outline frame
\begin{frame}{Outline}
    \tableofcontents
\end{frame}

%Question
\section{Problem Statement}
\begin{frame}{Problem Statement}
    \begin{block} {Papoulis Pillai Probability Random Variables and Stochastic Processes\\ 
    Exercise : 6-65}
    The random variables $X_i $ are i.i.d. and uniform in the interval (0,1). Show that if $ Y = max(X_i) $ then $F_Y(y) =y^n $ for $0 \leq y \leq 1$.
    
    \end{block}
\end{frame}
\section{Definitions}
\begin{frame}{Theory}
\begin{block}{}
If $X_1$ and $X_2$ are independent, then
\begin{align}
    P(X_1 \leq x , X_2 \leq x ) &= P(X_1\leq x) \cdot P( X_2 \leq x ) \label{1}
\end{align}
\end{block}
\begin{block}{}
A random variable $X$ is said to be uniformly distributed over an interval if the probability density function is given by
\begin{align}
    f_X(x)=&1 , \text{    for x in given interval} \label{2}
\end{align}
\end{block}
\begin{block}{}
$X_i$ are said to be i.i.d.(independent and identically distributed) if they are independent and have same pmf.
\end{block}
\end{frame}
\section{Solution}
\begin{frame}{Solution}
    Given,
    \begin{align}
        Y&= max(X_i)
        \implies \cbrak{Y \leq y} = \cbrak{X_1 \leq y,X_2 \leq y,\dots, X_n \leq y}
    \end{align}
Hence,
\begin{align}
    F_Y(y) &= P(Y \leq y)\\
            &= P(X_1 \leq y,X_2 \leq y,\dots, X_n \leq y) \label{5}
\end{align}
From \eqref{1} and \eqref{5},
\begin{align}
    F_Y(y)&=P(X_1 \leq y) P(X_2 \leq y) \dots  P(X_n \leq y)\\
    &=(P(X_1 \leq y))^n.......... (\text{Since } X_i \text{ are i.i.d})\\
    \implies F_Y(y) &=(F_{X_i}(y))^n \label{8}
\end{align}
\end{frame}

\begin{frame}{}
Given that $X_i$ are uniformly distributed over \{$0 \leq x \leq 1$\}\\
    From \eqref{2}
    \begin{align}
        f_{X_i}(x) &= 1 \\
        \implies F_{X_i}(x) &= x  & 0\leq x \leq 1
        \label{10}
    \end{align}
From \eqref{8} and \eqref{10},\\
For $ 0 \leq y \leq 1 $
\begin{align}
    F_Y(y) &=(F_{X_i}(y))^n=y^n\\
    \implies F_Y(y) &= y^n
\end{align}
\end{frame}
\end{document}